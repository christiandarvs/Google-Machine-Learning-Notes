\documentclass[a4paper,12pt]{article}

\usepackage{amsmath, amssymb, amsthm}
\usepackage{geometry}
\geometry{margin=1in}
\usepackage{graphicx}
\usepackage{hyperref}
\usepackage{xcolor}
\usepackage{enumitem}
\usepackage{listings}
\usepackage{float}
\usepackage{array}
\usepackage{placeins}
\usepackage{enumitem}
\usepackage{listings}
\usepackage{tikz}
\usepackage{amsmath}    
\usepackage{booktabs}
\usepackage{tabularx}

\title{What is Machine Learning?}
\author{Christian Darvin}
\date{\today}

\begin{document}

\maketitle

\section*{What is Machine Learning?}
Machine Learning (ML) gradually learns how to make predictions by studying lots of data.

\section*{Types of Machine Learning Systems}
\begin{enumerate}
    \item Supervised Learning
    \item Unsupervised Learning
    \item Reinforcement Learning
    \item Generative AI
\end{enumerate}

\section*{Supervised Learning}
The ML system is trained with the correct results. Imagine a student who is preparing for an upcoming exam. The student reviews old exams that contain both questions and answers in order to prepare for the new one.

\subsection*{Use Case 1: Regression (\#)}
A \textbf{regression model} predicts numerical values.

\begin{center}
\begin{tabularx}{\textwidth}{@{}lXl@{}}
\toprule
Scenario & Possible Input Data ($X$) & Predicted Number ($y$) \\
\midrule
Future Laptop Price & Operating System, CPU, RAM, GPU, Brand, Screen Size and Type & Price of the laptop \\
\bottomrule
\end{tabularx}
\end{center}

\subsection*{Use Case 2: Classification (?)}
A \textbf{classification model} predicts the likelihood that something belongs to a category. It returns a value indicating whether or not the item is in the category.

\subsubsection*{Binary Classification (0/1)}
Yields only two values: \texttt{[cake, not cake]}

\subsubsection*{Multiclass Classification (*)}
Yields more than two values: \texttt{[car, truck, motorcycle, bicycle, train]}

\section*{Unsupervised Learning}
The ML system isn't trained with the correct answers, nor does it contain any correct answers. It doesn't make predictions; instead, it finds meaningful patterns in the data.

\subsection*{Use Case 1: Clustering}
It finds data points that separate natural groupings and clusters those with similar features.

\section*{Reinforcement Learning}
The ML system makes predictions by obtaining \textbf{rewards} or \textbf{penalties} based on actions performed within an environment. It follows a \textbf{policy} which dictates the best strategy for getting the most rewards.

\section*{Generative AI}
The ML system returns content from the user input. It is \textit{multimodal}, capable of handling text, images, and video.

\begin{center}
\begin{tabular}{@{}ll@{}}
\toprule
Input & Output \\
\midrule
Text & Text \\
Text & Image \\
Text & Video \\
Text & Code \\
Image & Text \\
\bottomrule
\end{tabular}
\end{center}

The model is initially trained using an unsupervised learning approach, allowing it to learn patterns from the data. Afterward, a supervised or reinforcement learning approach is applied using data related to the model's intended capabilities.

\section*{References}
\begin{itemize}
    \item \href{https://developers.google.com/machine-learning/intro-to-ml/what-is-ml}{What is Machine Learning? - Google Developers}
    \item \href{https://developers.google.com/machine-learning/glossary}{Machine Learning Glossary - Google Developers}
\end{itemize}


\end{document}