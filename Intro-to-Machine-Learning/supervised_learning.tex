\documentclass[a4paper,12pt]{article}

\usepackage{amsmath, amssymb, amsthm}
\usepackage{geometry}
\geometry{margin=1in}
\usepackage{graphicx}
\usepackage{hyperref}
\usepackage{xcolor}
\usepackage{enumitem}
\usepackage{listings}
\usepackage{float}
\usepackage{array}
\usepackage{placeins}
\usepackage{enumitem}
\usepackage{listings}
\usepackage{tikz}
\usepackage{amsmath}    
\usepackage{booktabs}
\usepackage{tabularx}


\title{Supervised Learning}
\author{Christian Darvin}
\date{\today}

\begin{document}

\maketitle
\section*{Core Features}
\begin{enumerate}
    \item Data
    \item Model
    \item Training
    \item Evaluating
    \item Inference
\end{enumerate}

\section*{Data}
Datasets consist of individual entries that contain \textbf{features} $(X)$ and a \textbf{label}. It similar to a single row in a spreadsheet. \newline
\textbf{Features}: values that a supervised model uses to predict the label. \newline
\textbf{Label}: value that we want to predict. \newline

\begin{center}
\begin{tabularx}{\textwidth}{@{}lX@{}}
\toprule
\textbf{Feature} & \textbf{Value} \\
\midrule
Operating System & Windows 11 \\
CPU & Intel Core i7-12700H \\
RAM & 16GB DDR4 \\
GPU & NVIDIA RTX 3060 \\
Brand & ASUS \\
Screen Size and Type & 15.6" Full HD IPS \\
\textbf{Price (\$)} & \textbf{1,299} \\
\bottomrule
\end{tabularx}
\end{center}

\noindent Good datasets are both large (high in quantity) and highly diverse (covering a wide range of categories).


\section*{References}
\begin{itemize}
    \item \href{https://developers.google.com/machine-learning/intro-to-ml/supervised}{Supervised Learning - Google Developers}
    \item \href{https://developers.google.com/machine-learning/glossary}{Machine Learning Glossary - Google Developers}
\end{itemize}


\end{document}